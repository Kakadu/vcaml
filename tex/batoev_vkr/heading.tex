\usepackage{comment}

\newcommand{\authorTemplate}{kakadu}
%\newcommand{\authorTemplate}{daniil}

\usepackage{ifthen}
\ifthenelse{\equal{\authorTemplate}{kakadu}}
{% True case
  \newcommand{\result}{Template is Kakadu.}

  \usetheme{CambridgeUS}
  \usefonttheme{professionalfonts}
   
  % get rid of header navigation bar
  \setbeamertemplate{headline}{}
  % get rid of bottom navigation symbols
  \setbeamertemplate{navigation symbols}{}
  % get rid of footer
  %\setbeamertemplate{footline}{}
  %\setbeamertemplate{note page}[plain]
  %\setbeameroption{show notes on second screen=right}
  \addtobeamertemplate{title page}{}{
    \begin{center}{\tiny Build date: \today}\end{center}
  }
}
{% false case
  \newcommand{\result}{Template is NOT Kakadu.}
   
  \usetheme{CambridgeUS}
  \usefonttheme{professionalfonts}
     
   \usepackage{amssymb,amsmath,cancel,cite,color,cmap,float,graphicx,
     lmodern,listings,
     multirow,pifont,pgfplots,txfonts,tikz,wrapfig,xcolor,yfonts}
   \addtobeamertemplate{navigation symbols}{}{
     \usebeamerfont{footline}
     \fontsize{14pt}{14}\selectfont
     \usebeamercolor[fg]{footline}
     \hspace{1em}
     \circled{$\textswab{\insertframenumber}$}
   }
  }





\usepackage{xcolor}
\definecolor{YellowGreen} {HTML}{B5C28C}
\definecolor{ForestGreen} {HTML}{009B55}
\definecolor{MyBackground}{HTML}{F0EDAA}

\usepackage{xltxtra} % load xunicode
\usepackage{polyglossia}
\setmainlanguage{english}
\setotherlanguage{russian}
%\setmainlanguage{russian}
%\setotherlanguage{english}

%\let\cyrillicfonttt\monofamily
%\let\cyrillicfont\rmfamily
%\usepackage{fontspec}
%\newfontfamily{\cyrillicfont}{\rmfamily}

%\newfontfamily\dejaVuSansMono{DejaVu Sans Mono}
% https://github.com/vjpr/monaco-bold/raw/master/MonacoB/MonacoB.otf
%\newfontfamily\monacoB{MonacoB}

  
%%%%%%%%%%%%%%%%%%%%%%%%%%%%%%%%%%%%%%%%%%%%
\setmainfont[
 Ligatures=TeX,
 Extension=.otf,
 BoldFont=cmunbx,
 ItalicFont=cmunti,
 BoldItalicFont=cmunbi,
]{cmunrm}
% С засечками (для заголовков)
\setsansfont[
 Ligatures=TeX,
 Extension=.otf,
 BoldFont=cmunsx,
 ItalicFont=cmunsi,
]{cmunss}
\setmonofont{Latin Modern Mono}

\usefonttheme{professionalfonts}

%%%%%%%%%%%%%%%%%%%%%%%%%%%%%%%%%%%%%%%%%%%%
\usepackage{listings}
\lstdefinelanguage{ocanren}{
keywords={run, conde, fresh, let, in, match, with, when, class, type,
object, method, of, rec, repeat, until, while, not, do, done, as, val, inherit,
new, module, sig, deriving, datatype, struct, if, then, else, open, private, virtual, include, success, failure,
true, false},
sensitive=true,
commentstyle=\small\itshape\ttfamily,
keywordstyle=\ttfamily\textbf,
identifierstyle=\ttfamily,
basewidth={0.5em,0.5em},
columns=fixed,
mathescape=true,
fontadjust=true,
literate={fun}{{$\lambda$}}1 {->}{{$\to$}}3 {===}{{$\equiv$}}1 {=/=}{{$\not\equiv$}}1 {|>}{{$\triangleright$}}3 {\\/}{{$\vee$}}2 {/\\}{{$\wedge$}}2 {^}{{$\uparrow$}}1,
morecomment=[s]{(*}{*)}
}

\lstset{
language=ocanren
}
%%%%%%%%%%%%%%%%%%%%%%%%%%%%%%%%%%%%%%%%%%%%

\usepackage{times}


\usepackage{amsmath}
%\DeclareMathOperator{->}{\rightarrow}
%\newcommand\iso{\ensuremath{\cong}}
\newcommand{\Haskell}{\textsc{Haskell}}
\newcommand{\haskell}{\Haskell}
\newcommand{\OCaml}{\textsc{OCaml}}
\newcommand{\miniKanren}{\textsc{miniKanren}}
\newcommand{\OCanren}{\textsc{OCanren}}
\newcommand{\noCanren}{\textsc{noCanren}}
\newcommand{\newln}{\vspace{1em}}


\usepackage{subcaption}   % for subfigures
\usepackage{verbatim}
\usepackage{graphicx}   % for \includegraphics{file.pdf}

\usepackage{fontawesome}
%\newfontfamily{\FAPro}{Font Awesome 5 Pro}
\expandafter\def\csname faicon@facebook\endcsname{{\FA\symbol{"F09A}}}
\def\faQuestionSign{{\FA\symbol{"F059}}}
\def\faQuestion{{\FA\symbol{"F128}}}
\def\faExclamation{{\FA\symbol{"F12A}}}
\def\faUploadAlt{{\FA\symbol{"F093}}}
\def\faLemon{{\FA\symbol{"F094}}}
\def\faPhone{{\FA\symbol{"F095}}}
\def\faCheckEmpty{{\FA\symbol{"F096}}}
\def\faBookmarkEmpty{{\FA\symbol{"F097}}}

%\def\faSadTear{{\FAPro\symbol{"F5B4}}}
%\def\faMeh{{\FAPro\symbol{"F11A}}}
%\def\faGrinStars{{\FAPro\symbol{"F587}}}

\newcommand{\faGood}{\textcolor{ForestGreen}{\faThumbsUp}}
\newcommand{\faBad}{\textcolor{red}{\faThumbsODown}}
\newcommand{\faWrong}{\textcolor{red}{\faTimes}}
\newcommand{\faMaybe}{\textcolor{blue}{\faQuestion}}
\newcommand{\faCheckGreen}{\textcolor{ForestGreen}{\faCheck}}


% sudo aptget install ttf-mscorefonts-installer
\defaultfontfeatures{Ligatures={TeX}}
\setmainfont{Times New Roman}
\setsansfont{CMU Sans Serif}

\setmonofont[Scale=1.0,
    BoldFont=lmmonolt10-bold.otf,
    ItalicFont=lmmono10-italic.otf,
    BoldItalicFont=lmmonoproplt10-boldoblique.otf
]{lmmono9-regular.otf}

\usepackage[cache=true]{minted}

%\usepackage{inconsolata}

%\newfontfamily{\monott}{Latin Modern Mono}

\usepackage{listings}
\lstdefinelanguage{ocanren}{
keywords={run, conde, fresh, let, in, match, with, when, class, type,
object, method, of, rec, repeat, until, while, not, do, done, as, val, inherit,
new, module, sig, deriving, datatype, struct, if, then, else, open, private, virtual, include, success, failure,
true, false},
sensitive=true,
commentstyle=\small\itshape\ttfamily
,keywordstyle=\fontfamily{monott}\selectfont \textbf
,identifierstyle=\fontfamily{monott}\selectfont%\ttfamily
,basewidth={0.5em,0.5em},
columns=fixed,
mathescape=true,
fontadjust=true,
literate={fun}{{$\lambda$}}1 {->}{{$\to$}}3 {===}{{$\equiv$}}1 {=/=}{{$\not\equiv$}}1 {|>}{{$\triangleright$}}3 {\\/}{{$\vee$}}2 {/\\}{{$\wedge$}}2 {^}{{$\uparrow$}}1,
morecomment=[s]{(*}{*)}
}

\setmonofont[Mapping=tex-text]{CMU Typewriter Text}
\lstdefinelanguage{ocaml}{
keywords={@type, function, fun, let, in, match, with, when, class, type,
nonrec, object, method, of, rec, repeat, until, while, not, do, done, as, val, inherit, and,
new, module, sig, deriving, datatype, struct, if, then, else, open, private, virtual, include, success, failure, switch,
lazy, assert, true, false, end},
sensitive=true,
commentstyle=\small\itshape\ttfamily,
keywordstyle=\ttfamily\bfseries, %\underbar,
%identifierstyle=\fontfamily{cmtt}\selectfont\ttfamily,
identifierstyle=\ttfamily,
basewidth={0.5em,0.5em},
columns=fixed,
fontadjust=true,
literate={->}{{$\to$}}3 {===}{{$\equiv$}}1 {=/=}{{$\not\equiv$}}1 {|>}{{$\triangleright$}}3 {\\/}{{$\vee$}}2 {/\\}{{$\wedge$}}2 {>=}{{$\ge$}}1 {<=}{{$\le$}} 1,
morecomment=[s]{(*}{*)}
}

\def\HaskellTypeclassColor\PYG{k+kt}
\def\HaskellCommentColor\PYG{c+c1}
% TODO: https://tex.stackexchange.com/questions/4198/emphasize-word-beginning-with-uppercase-letters-in-code-with-lstlisting-package
\lstdefinelanguage{haskell}{
basicstyle=\large\ttfamily,   % Вот тут надо стиль ставить, а не у идентификаторов
identifierstyle=\ttfamily,
commentstyle=\HaskellCommentColor\itshape\HaskellCommentColor,
sensitive=true,
%
classoffset=0
  , keywords={where, let, in, when, class, type, data, of, do, as, val, inherit, module, sig, deriving, if, then, else, import , assert, true, false, end}
  , keywordstyle=\ttfamily\bfseries\color{ForestGreen} %\underbar
, classoffset=1
  , morekeywords={pure,empty,select,branch,oneOf}
  , keywordstyle=\color{RawSienna},
classoffset=2
  , morekeywords={Monad,Applicative,Selective,String
      ,Either,Left,Right
      ,Maybe,Some,None
      }
  , keywordstyle=\PYG{k+kt}
, classoffset=0,
%keywordstyle=[2]{\color{orange}},
otherkeywords={::,<$>, >?>},
%identifierstyle=\fontfamily{cmtt}\selectfont\ttfamily,
%basewidth={0.5em,0.5em},
columns=fixed,
%fontadjust=true,
%literate={->}{{$\to$}}3 {===}{{$\equiv$}}1 {=/=}{{$\not\equiv$}}1 {|>}{{$\triangleright$}}3 {\\/}{{$\vee$}}2 {/\\}{{$\wedge$}}2 {>=}{{$\ge$}}1 {<=}{{$\le$}} 1,
%morecomment=[s]{(*}{*)}
%, literate={\$}{{\textcolor{blue}{\$}}}1
%, literate={<\$>}{{\textcolor{RawSienna}{\ <\$>\ } }}1
%           {>?>}{{\textcolor{RawSienna}{\ >?>\ } }}1
}
% Для перевода теорем на русский
\patchcmd{\definition}{Definition}{Определение}{}{}

\definecolor{light-gray}{gray}{0.80}


\usepackage{soul} % for \st that strikes through
\usepackage[normalem]{ulem} % \sout
\usepackage{tabulary}   % for \begin{tabular}{c|cp{2cm}cccc}, etc.
\let\emptyset\varnothing
\let\eps\varepsilon

%%%% 
\usepackage{tikz}
\usetikzlibrary{cd}
\usetikzlibrary{decorations.pathreplacing,calc,shapes,positioning,tikzmark}
%\usetikzlibrary{arrows,shapes}

% See also 
% https://tex.stackexchange.com/questions/284165
% https://tex.stackexchange.com/questions/284311
\newcounter{tmkcount}

\tikzset{
 use tikzmark/.style={
   remember picture,
   overlay,
   execute at end picture={
     \stepcounter{tmkcount}
   },
 },
 tikzmark suffix={-\thetmkcount}
}
% %%%%%%%%%%%%%%%%%%%%%%%%%%%%%%%%%%%%%%%%%%%%%%%%%%
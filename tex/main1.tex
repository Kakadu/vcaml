\documentclass{article}
\usepackage[cache=true]{minted}
\usepackage{polyglossia}
\usepackage{comment}

\usepackage[ left=1cm
           , right=2cm
           , top=1cm
           , bottom=1.5cm
           ]{geometry}
\usepackage{amssymb,amsmath,amsthm,amsfonts} 

\usepackage{fontspec}
% \DeclareMathSizes{22}{30}{24}{20}


%\setromanfont[
%    BoldFont=timesbd.ttf,
%    ItalicFont=timesi.ttf,
%    BoldItalicFont=timesbi.ttf,
%    Scale=1.5
%]{times.ttf}
%\setsansfont[
%    BoldFont=arialbd.ttf,
%    ItalicFont=ariali.ttf,
%    BoldItalicFont=arialbi.ttf,
%    Scale=1.5
%]{arial.ttf} 
%\setmonofont[Scale=0.90,
%    BoldFont=courbd.ttf,
%    ItalicFont=couri.ttf,
%    BoldItalicFont=courbi.ttf,
%    Scale=1.5
%]{cour.ttf}

\defaultfontfeatures{Ligatures={TeX}}
\setmainfont[Scale=1.5]{Times New Roman}
\setsansfont[Scale=1.5]{CMU Sans Serif}

\setmonofont[ BoldFont=lmmonolt10-bold.otf
			, ItalicFont=lmmono10-italic.otf
			, BoldItalicFont=lmmonoproplt10-boldoblique.otf
			, Scale=1.5
]{lmmono9-regular.otf}

\usepackage{unicode-math}
\setmathfont{Latin Modern Math}[Scale=MatchLowercase]
\newcommand*{\arr}{\ensuremath{\rightarrow}}
\newcommand{\dd}[1]{\mathrm{d}#1}
\newcommand{\effect}[1]{\text{effect}_{#1}}
\newcommand{\result}[1]{\text{result}_{#1}}
\newcommand{\find}[3]{\ensuremath{\texttt{find}(#1, #2,#3)}}

\makeatletter
\def\li{\@ifnextchar[{\@with}{\@without}}
\def\@with[#1]#2{\ensuremath{ \texttt{LI}(#1, #2) }}
\def\@without#1{\ensuremath{ \texttt{LI}(#1) }}
\makeatother


% Doesn't work?
\renewcommand{\epsilon}{\ensuremath{\varepsilon}}
%\renewcommand{\sigma}{\ensuremath{\varsigma}}

\begin{document}

\begin{minted}[mathescape, escapeinside=??]{ocaml }  
val encode: term -> heap -> term  (* ??? *)

let encode term ?$\sigma$? = 
  gmap term (object 
                method LI h x ?$\arr$? find h x ?$\sigma$? 
                method Call f ?$\arr$? encode (result_of f) ?$\sigma$? 
            end)

let find heap x ?$\sigma$? =
  match heap with 
  | Defined hs  ?$\arr$? find_defined hs x ?$\sigma$? 
  | Merge [<?$g_i$?, ?$h_i$?>; ...] ?$\arr$? 
     UNION [ <encode ?$g_i$? ?$\sigma$?, find ?$h_i$? x ?$\sigma$?>; ...]
  | RecApp f  ?$\arr$? find (effect_of f) x ?$\sigma$?
  | h1 ?$\circ$? h2   ?$\arr$? find h2 x (fun x -> find h1 x ?$\sigma$?)
\end{minted}

\section{Примеры}
\subsection{Пример из статьи}

\textbf{N.B.} В статье такой язык, где функции не возвращают содержательного результата, т.е. имеют тип \mintinline{ocaml}{unit}. Иными словами, содержательное поведение функции -- это её эффект на глобальном состояии. Наверное можно немного изменить, чтобы поддержать содержательные результаты функции, но пока хз как.

Используется функция $SAT(Env, Rec(f), Phormula)$, которая означает что в контексте результатов символьного исполнения $Env$ после запуска функции $f$ не случается $Assert(Phormula)$, иными словами, после исполнения $f$ на произвольных аргументах формула не выполнима.

Там есть типа алгоритм
\begin{enumerate}
  \item Нормализация $e$ и преобразование трёх верхнеуровневого объединения в конструкцию ветвления.
  \item Замена ячеек $LI(\sigma,x)$  на  $find(\sigma, x, \tau)$.
  \item Специализация $find$ согласно правилам.
\end{enumerate}

Например, надо проверить $SAT(Env, Rec(f), LI(\varepsilon, a) * 3< 17)$ при $Body(f) = $
$merge(<c,\varepsilon>, <\neg c, \sigma \circ Rec(f)>)$, т.е. ограничение 
$g = (Rec(f)\bullet LI(\varepsilon,a)*3<17) = (LI(Rec(f), a)*3 < 17)$ при некоторых наперед заданных $a$, $c$ и $\sigma$.

Контекст, в котором происходит первый вызов $f$ будем обозначать $\tau_0$.

Избавляемся от $LI$: $g = (find(Rec(f), a, \tau_0)*3 < 17)$

Третий шаг порождает специализацию $find_f$, а закодированная формула становится 
$g = (find_f\ \tau_0\ a)*3 < 17$.

Теперь надо описать функцию $find_f\ \tau\ x = ... $ Тело получается применением шагов 1-3 к $Body(f)$: 

\begin{center}
$find( merge(<c,\varepsilon>, <\neg c, \sigma \circ Rec(f)>),\ x,\ \tau)$ = 
 $ite(\tau\bullet c,\ find(\varepsilon,x,\tau),\ find(\sigma\circ Rec(f),x,\tau)$
\end{center}
Здесь получаются два $find$, которые надо специализировать.

$find_{\varepsilon} \tau\ x = \tau\ x$

$find_{\sigma\circ\ Rec(f)}\ \tau\ x = find_{f}\ (find_{\sigma}\ \tau)\ x$

Функция $find_\sigma$ задана свыше, так как $\sigma$ задана свыше.

\subsection{Фибоначчи самый наивный}
\begin{minted}{ocaml}  
let rec fib1 n = 
  if n <=1 then 1
  else if n <= 2 then 1
  else (fib1 (n-1)) + (fib1 (n-2))
\end{minted}

После символьного исполнения получается пустой эффект и терм:
\begin{minted}[mathescape, escapeinside=??]{ocaml}
effect_of(?$fib_{1002}$?) = ?$\varepsilon$?
result_of(?$fib_{1002}$?) = 
  Union 
    [ ?$(n_{1003} \leqslant 1, \quad 1) $?
    ; ?$(n_{1003} > 1 \wedge n_{1003} \leqslant 2,\quad 1)$?
    ; ?$(n_{1003} > 1 \wedge n_{1003} > 2$?
      , (Call (Ident ?$fib_{1002}$?) ?$\{n_{1003} \mapsto n_{1003} - 2\}$? +
        (Call (Ident ?$fib_{1002}$?) ?$\{n_{1003} \mapsto n_{1003} - 2\}  )$?   
    ]
\end{minted}
Упрощение
\begin{minted}[mathescape, escapeinside=??]{ocaml}
effect_of(?$fib_{1002}$?) = ?$\varepsilon$?
result_of(?$fib_{1002}$?) = 
  Union 
    [ ?$(n_{1003} \leqslant 2, \quad 1)$?
    ; ?$(n_{1003} > 2 $?
      , (Call (Ident ?$fib_{1002}$?) ?$\{n_{1003} \mapsto n_{1003} - 2\}$?) +
        (Call (Ident ?$fib_{1002}$?) ?$\{n_{1003} \mapsto n_{1003} - 2\} $?) ?$)$?

    ]
\end{minted}
Теперь свойство, от которого будем доказывать: 
\begin{minted}[mathescape, escapeinside=??]{ocaml}
?$\forall n . (n \geqslant 1) \Rightarrow$? (Call (Ident ?$fib_{1002}$?) ?$\{n_{1003} \mapsto n\}$?) ?$\geqslant n$?
\end{minted}

Сейчас хотим специализировать \mintinline{latex}{find} для \mintinline{latex}{fib1}, избавляясь по дороге от композиций куч и LI.

\begin{minted}[mathescape, escapeinside=??]{ocaml}
encode ((Call (Ident ?$fib_{1002}$?) ?$\{n_{1003}\ \mapsto n\}$?))), ?$\sigma$?)
encode ( (result_of ?$fib_{1002}$?) ?$\{n_{1003}\ \mapsto n\}$?})) ), ?$\sigma$?)

let encoded_fib ?$\{ n_{1003}\ \mapsto\ n_{1003}\}$? = 
    if ?$n_{1003} \leqslant 2$? then ?$1$?
    else 
        encoded_fib ?$\{ n_{1003}\ \mapsto\ (n_{1003} - 1)\}$? +  
        encoded_fib ?$\{ n_{1003}\ \mapsto\ (n_{1003} - 2)\}$?
\end{minted}


% \begin{minted}[mathescape, escapeinside=??]{ocaml}
% ?$\forall n . (n \geqslant 1) \Rightarrow$? find (Call (LI ("fib1_1002")), ?$n$?, WTF) ?$\geqslant n$?
% 
% ?$\forall n . (n \geqslant 1) \Rightarrow$? find (effect_of("fib1_1002"), ?$n$?, WTF) ?$\geqslant n$?
% 
% ?$\forall n . (n \geqslant 1) \Rightarrow$? find (?$\varepsilon$?, ?$n$?, WTF) ?$\geqslant n$?
% 
% ?$\forall n . (n \geqslant 1) \Rightarrow$? (find_defined ?$\varepsilon$?  ?$n$? WTF) ?$\geqslant n$?
% \end{minted}

\begin{comment}
\subsection{Фибоначчи императивный (попытка 1)}
\begin{minted}{ocaml}  
let a = ref 0
let b = ref 0

let rec loop n = 
  if n <= 1
  then ()
  else 
    let c  = !a + !b in 
    let () = a := !b in
    b := c;
    loop (n-1) 
 
let fib ndx = 
  a := 1;
  b := 1;
  loop ndx;
  !b

[@@@assert (forall x (fib m >= m))]

\end{minted}
% Может для полноты ещё написать для a и b
После символьного исполнения $loop_{1635}$ и упрощения $a \neq b \wedge c \neq a \wedge c \neq b$
\begin{minted}[mathescape, escapeinside=??]{ocaml}
effect_of(?$loop_{1635}$?) = HMerge 
  [ (?$n_{1636} \leqslant 1$?, ?$\varepsilon$?)
  ; (?$n_{1636} > 1$?,
      [ (?$b_{1634} \mapsto (a_{1002} + b_{1634})$?) (* write LIs here *)
      ; (?$c_{1637} \mapsto (a_{1002} + b_{1634})$?)
      ; (?$a_{1002} \mapsto b_{1634}$?)          
      ] ?$\circ$? (RecApp ?$loop_{1635}$? ?$\{n_{1636} \mapsto n_{1636} - 1\}$?)
    )
  ]
result_of(?$loop_{1635}$?) = Union 
  [ ( ?$n_{1636} \leqslant$? 1, Unit)     
  ; ( ?$n_{1636} >$? 1,
      Call (Ident ?$loop_{1635}$?) ?$\{n_{1003} \mapsto (n_{1003} - 1)\}$?) )
  ]
\end{minted}
После символьного исполнения $fib_{1638}$
\begin{minted}[mathescape, escapeinside=??]{ocaml}
effect_of(?$fib_{1638}$?) = 
  ?$\{ b_{1634} \mapsto 1;\quad a_{1002} \mapsto 1\} \ \circ$? (RecApp (Ident ?$ loop_{1635}) \ \{ n_{1636} \mapsto ndx_{1639} \}$?)
result_of(?$fib_{1638}$?) = ?$b_{1634}$? 
\end{minted}

Теперь свойство, от которого будем доказывать: 
\begin{minted}[mathescape, escapeinside=??]{ocaml}
?$\forall n . (n \geqslant 1) \Rightarrow$? (Call (Ident ?$fib_{1638}$?) ?$\{n_{1003} \mapsto ndx_{1639}\}$?) ?$\geqslant n$?
\end{minted}

В терминах функции SAT: $SAT(Body, Rec(loop_{1635}), b_{1638} \leqslant n_{1003})$. Другими словами, после выполнения функции $loop_{1635}$  на куче $\tau_0=\{a\mapsto 1, b\mapsto 1, ndx\mapsto n\}$ (в ней закодирован в том числе аргумент) формула $b_{1638} \leqslant n_{1003}$ будет невыполнима.

Поехали. Нужно проверять $g = (Rec(loop)\bullet LI(\varepsilon, b) > n)$.

Пихаем композицию внутрь:  $(LI(Rec(loop), b) > n)$.

Вводим find: $find(Rec(loop), b, \tau_0) > n$.

Порождаем специализацию $find_{loop}\ \tau\ x = ...$ и получаем формулу $$
find_{loop}\  \tau_0\ b > n
$$
$$
\tau_0=\{a\mapsto 1, b\mapsto 1, ndx\mapsto n\}
$$

Теперь надо специализировать тело функции $loop$...

\end{comment}

\subsection{Фибоначчи императивный (попытка 2)}
\begin{minted}{ocaml}  
let a = ref 0
let b = ref 0

let rec loop n = 
  if n <= 1
  then ()
  else 
    let c  = !a + !b in 
    let () = a := !b in
    b := c;
    loop (n-1) 
 
let fib ndx = 
  a := 1;
  b := 1;
  loop ndx;
  !b

[@@@assert (forall m (fib m >= m))]

\end{minted}
% Может для полноты ещё написать для a и b
После символьного исполнения $loop_{1635}$ и упрощения $a \neq b \wedge c \neq a \wedge c \neq b$


\begin{equation}
\sigma_0 = \begin{cases}
a \mapsto 1\\ 
b \mapsto 1\\ 
n \mapsto \li{ndx}
\end{cases}
\end{equation}

\begin{equation}
\effect{fib} = \sigma_0 \circ loop
\end{equation}

\begin{equation}
\effect{loop} = Merge \begin{cases}
\varepsilon& \text{если } \li{n} \leq 1 \\
\sigma \circ loop & \text{если } \li{n} > 1 
\end{cases}
\end{equation}

\begin{equation}
\sigma = \begin{cases}
c \ \mapsto \ \li{a} + \li{b} \\ 
a \ \mapsto \ \li{b} \\ 
b \ \mapsto \ \li{a} + \li{b} \\ 
n \ \mapsto \ \li{n} -1
\end{cases}
\end{equation}
Терм-результат для $fib$ -- это значение в локации $b$ после совершения над глобальных состоянием эффекта $\sigma_0$ и эффекта вызова функции $loop$\footnote{У нас тут инициализация аргумента функции склеилась с глобальным состоянием на момент вызова, но это не существенно}.
\begin{align*}
\result{fib} &= \li{\sigma_0\circ loop,b}\\
\result{loop} &= \texttt{Unit}
\end{align*}
Теперь посмотрим на инвариант, который хотим проверить. Инициализация аргумента будет сделано просто:
\begin{equation}
\sigma_1 = \{ ndx \mapsto \li{m} \}
\end{equation}
А сам инвариант будет выглядеть как
\begin{equation}
\sigma_1 \circ \result{fib} \ge \li{m}
\end{equation}
Подставив $ \result{fib}$ и применив "дистрибутивность" получим:
\begin{align}
\sigma_1 \circ \result{fib}  &\ge \li{m} \\
\sigma_1 \circ \li{\sigma_0\circ loop,b}  &\ge \li{m} \\
 \li{\sigma_1 \circ \sigma_0\circ loop,b}  &\ge \li{m} \\
\end{align}
Преобразуем все $\li{*,*}$ в $find$ используя в качестве контекстной кучи $\tau$:
\begin{align}
\find{\sigma_1 \circ \sigma_0\circ loop}{b}{\tau}  &\ge \find{\varepsilon}{m}{\tau}
\end{align}
что порождает следующий OCaml код
\begin{align}
\texttt{find}_{\sigma_1 \circ \sigma_0\circ loop}\;\tau\; b  &\ge \texttt{find}_{\varepsilon} \tau\; m
\end{align}
В общем результирующий OCaml код будет состоять из различных специализаций функции $\texttt{find}_{heap}$ для различных куч. Самый простой случай -- специализация для пустой кучи -- делегирует работу контекстной куче
\begin{align}
\texttt{find}_{\varepsilon} \tau\; x = \tau\; x
\end{align}
Все сгенерированные функции первого порядка $find_{heap}$ принимают контекстную кучу и имя локации и выдают значение.

Предъявим специализации для более сложных случаев.
\begin{align}
\texttt{find}_{\sigma_1 \circ \sigma_0\circ loop}\; \tau \; x  &= 
\texttt{find}_{loop}\; (\texttt{find}_{\sigma_1 \circ \sigma_0} \tau) \; x\\
\texttt{find}_{\sigma_1 \circ \sigma_0}\; \tau \; x  &= 
\texttt{find}_{\sigma_0}\; (\texttt{find}_{\sigma_1} \tau) \; x
\end{align}

Специализации $find$ для конкретных куч почти повторяют структуру соответствующих конкретных куч
\begin{align}
\texttt{find}_{\sigma_0}\; \tau \; x  = \begin{cases}
1& \text{если } x \sim "a"\\
1& \text{если } x \sim "b"\\
\texttt{find}_{\varepsilon}\; \tau\; ndx& \text{если } x \sim "n"\\
\tau\  x& \text{иначе}
\end{cases}\\
\texttt{find}_{\sigma_1}\; \tau \; x  = \begin{cases}
\texttt{find}_{\varepsilon}\; \tau\; m& \text{если } x \sim "ndx"\\
\tau\  x& \text{иначе}
\end{cases}
\end{align}

\begin{align}
\texttt{find}_{loop}\; \tau \; x  = \begin{cases}
\texttt{find}_{\varepsilon} \tau x & \text{если } \texttt{find}_{\varepsilon} n \le 1\\
\texttt{find}_{loop} (\texttt{find}_{\sigma}\; \tau) x & \text{если } \texttt{find}_{\varepsilon} n > 1\\
\texttt{raise Unreachable}& \text{иначе}
\end{cases}
\end{align}


%%%%%%%%%%%%%%%%%%%%%%%%%%%%%%%%%%%%%%%%%%%%%
\subsection{Ещё какой-то тест}
\begin{minted}{ocaml}  
let a = ref 0

let loop n = 
  !a
 
[@@@assert (forall m (loop m >= 0))]

\end{minted}


\begin{equation}
\sigma_0 = \begin{cases}
a \mapsto 1\\ 
b \mapsto 1\\ 
n \mapsto \li{ndx}
\end{cases}
\end{equation}

\begin{equation}
\effect{fib} = \sigma_0 \circ loop
\end{equation}

\begin{equation}
\effect{loop} = Merge \begin{cases}
\varepsilon& \text{если } \li{n} \leq 1 \\
\sigma \circ loop & \text{если } \li{n} > 1 
\end{cases}
\end{equation}

\begin{equation}
\sigma = \begin{cases}
c \ \mapsto \ \li{a} + \li{b} \\ 
a \ \mapsto \ \li{b} \\ 
b \ \mapsto \ \li{a} + \li{b} \\ 
n \ \mapsto \ \li{n} -1
\end{cases}
\end{equation}
Терм-результат для $fib$ -- это значение в локации $b$ после совершения над глобальных состоянием эффекта $\sigma_0$ и эффекта вызова функции $loop$\footnote{У нас тут инициализация аргумента функции склеилась с глобальным состоянием на момент вызова, но это не существенно}.
\begin{align*}
\result{fib} &= \li{\sigma_0\circ loop,b}\\
\result{loop} &= \texttt{Unit}
\end{align*}
Теперь посмотрим на инвариант, который хотим проверить. Инициализация аргумента будет сделано просто:
\begin{equation}
\sigma_1 = \{ ndx \mapsto \li{m} \}
\end{equation}
А сам инвариант будет выглядеть как
\begin{equation}
\sigma_1 \circ \result{fib} \ge \li{m}
\end{equation}
Подставив $ \result{fib}$ и применив "дистрибутивность" получим:
\begin{align}
\sigma_1 \circ \result{fib}  &\ge \li{m} \\
\sigma_1 \circ \li{\sigma_0\circ loop,b}  &\ge \li{m} \\
 \li{\sigma_1 \circ \sigma_0\circ loop,b}  &\ge \li{m} \\
\end{align}
Преобразуем все $\li{*,*}$ в $find$ используя в качестве контекстной кучи $\tau$:
\begin{align}
\find{\sigma_1 \circ \sigma_0\circ loop}{b}{\tau}  &\ge \find{\varepsilon}{m}{\tau}
\end{align}
что порождает следующий OCaml код
\begin{align}
\texttt{find}_{\sigma_1 \circ \sigma_0\circ loop}\;\tau\; b  &\ge \texttt{find}_{\varepsilon} \tau\; m
\end{align}
В общем результирующий OCaml код будет состоять из различных специализаций функции $\texttt{find}_{heap}$ для различных куч. Самый простой случай -- специализация для пустой кучи -- делегирует работу контекстной куче
\begin{align}
\texttt{find}_{\varepsilon} \tau\; x = \tau\; x
\end{align}
Все сгенерированные функции первого порядка $find_{heap}$ принимают контекстную кучу и имя локации и выдают значение.

Предъявим специализации для более сложных случаев.
\begin{align}
\texttt{find}_{\sigma_1 \circ \sigma_0\circ loop}\; \tau \; x  &= 
\texttt{find}_{loop}\; (\texttt{find}_{\sigma_1 \circ \sigma_0} \tau) \; x\\
\texttt{find}_{\sigma_1 \circ \sigma_0}\; \tau \; x  &= 
\texttt{find}_{\sigma_0}\; (\texttt{find}_{\sigma_1} \tau) \; x
\end{align}

Специализации $find$ для конкретных куч почти повторяют структуру соответствующих конкретных куч
\begin{align}
\texttt{find}_{\sigma_0}\; \tau \; x  = \begin{cases}
1& \text{если } x \sim "a"\\
1& \text{если } x \sim "b"\\
\texttt{find}_{\varepsilon}\; \tau\; ndx& \text{если } x \sim "n"\\
\tau\  x& \text{иначе}
\end{cases}\\
\texttt{find}_{\sigma_1}\; \tau \; x  = \begin{cases}
\texttt{find}_{\varepsilon}\; \tau\; m& \text{если } x \sim "ndx"\\
\tau\  x& \text{иначе}
\end{cases}
\end{align}

\begin{align}
\texttt{find}_{loop}\; \tau \; x  = \begin{cases}
\texttt{find}_{\varepsilon} \tau x & \text{если } \texttt{find}_{\varepsilon} n \le 1\\
\texttt{find}_{loop} (\texttt{find}_{\sigma}\; \tau) x & \text{если } \texttt{find}_{\varepsilon} n > 1\\
\texttt{raise Unreachable}& \text{иначе}
\end{cases}
\end{align}


\end{document}
